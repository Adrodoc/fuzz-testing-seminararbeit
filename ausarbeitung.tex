\documentclass[a4paper,11pt]{article}

% Macht evtl unter Windows Probleme?
\usepackage[utf8]{inputenc}

% weitere Pakete hier
\usepackage{amsmath}
\usepackage{multicol}
\usepackage{xcolor}

\definecolor{comment_color}{RGB}{63,127,89}
\definecolor{keyword_color}{RGB}{127,0,85}
\definecolor{string_color}{RGB}{42,0,255}
\usepackage{listings}
\lstset{language=Java,
  basicstyle=\ttfamily,
  commentstyle=\color{comment_color},
  keywordstyle=\color{keyword_color},
  showstringspaces=false,
  stringstyle=\color{string_color},
}

% Referenzen schön formatieren, Kommandos
% \cite  -> [x]
% \citet -> Name [x]
\usepackage[sort&compress,numbers]{natbib}
\setlength{\bibsep}{2pt plus 0.3ex}
\renewcommand*{\bibfont}{\small}
\makeatletter
\def\NAT@spacechar{~}% NEW
\makeatother

% Klickbare Links im Dokument
\usepackage{hyperref} % muss als vorletztes Paket eingebunden werden
% Evtl auskommentieren

% Querverweise mit \cref und \Cref
\usepackage[ngerman]{babel}
\usepackage[ngerman,noabbrev]{cleveref} % muss als letztes Paket eingebunden werden


%%%%%%%%%%%%%%%%%%%%%%%%%%%%%%%%%%%%%%%%
\title{\textbf{EvoSuite}\\
Automatic Test Suite Generation for Java}
\author{Adrian Uffmann\\
Matrikelnummer: 12043921\\
\texttt{adrian.uffmann@campus.lmu.de}}
\date{09.02.2020}
%%%%%%%%%%%%%%%%%%%%%%%%%%%%%%%%%%%%%%%%

\begin{document}
\maketitle

\begin{center}
Masterseminar Fuzz Testing
\end{center}

\begin{abstract}
EvoSuite ist ein Programm, welches mithilfe eines genetischen Algorithmus automatisiert JUnit-Testfälle aus Java-Bytecode generieren kann.
Auf der Internetseite des Projektes (\url{http://www.evosuite.org/publications/}) werden 54 Publikationen zu EvoSuite und dessen Vorgänger ${\mu}Test$ aufgelistet.
Diese Seminararbeit gibt einen Überblick über die Funktionsweise von EvoSuite.
\end{abstract}

\section{Einleitung}

Das Testen von Software ist ein wichtiger Teil der Softwareentwicklung, so schreiben \citet{myers2004art}:
\begin{quote}
it [is] a well-known rule of thumb that in a typical programming project approximately 50 percent of the elapsed time and more than 50 percent of the total cost [are] expended in testing the program or system being developed.
\end{quote}
Daher ist es kaum verwunderlich, dass es zahlreiche Werkzeuge gibt, die helfen sollen das Testen von Software zu automatisieren.
EvoSuite geht noch einen Schritt weiter: hier wird nicht versucht das Testen der Software zu automatisieren, sondern das Erstellen von automatisierten Tests selbst durch Automatisierung zu vereinfachen.
Der Fokus liegt dabei auf der Optimierung eines \textit{test last} Ansatzes, d. h. getestet wird erst, nachdem das eigentliche Programm fertig implementiert ist.
Bei diesem Ansatz würde sich ein Softwaretester zunächst Szenarien überlegen, mit denen er möglichst viele Pfade des Programms abdeckt und diese als Unittests ausformulieren.
Mithilfe von sogenannten Orakeln wird dabei ein bestimmtes Verhalten des Programms definiert.
Verhält sich das Programm anders als von den Orakeln vorgegeben, so schlägt der Test fehl und es wurde ein Bug gefunden.

EvoSuite versucht nun diesen Prozess zu vereinfachen, indem es Unittests generiert, die das Verhalten des Programms möglichst gut beschreiben.
Ein Softwaretester muss sich dann nicht mehr selbst Szenarien überlegen, sondern nur noch die generierten Testfälle auf unerwartetes Verhalten prüfen.
Dazu nutzt EvoSuite einen genetischen Algorithmus mit dem zufällige Test-fälle immer weiter verbessert werden, bis sie eine möglichst hohe Testabdeckung erreichen.
Die Testfälle werden dann minimiert um die Arbeit des Softwaretesters zu vereinfachen und es werden mit sogenanntem \textit{Mutation-Testing} möglichst repräsentative Orakel ausgewählt.

EvoSuite wurde seit 2010 immer weiter entwickelt und es sind verschiedene Algorithmen für das generieren von Testfällen implementiert, die über die Kommandozeile ein- und ausgeschaltet werden können.
Dazu haben die Autoren von EvoSuite auf der Internetseite (\url{http://www.evosuite.org/publications/}) 54 Artikel über die Konzepte in und die Evaluation von EvoSuite und dessen Vorgänger ${\mu}Test$ veröffentlicht.
Um den vorgegebenen Umfang einzuhalten, kann diese Seminararbeit nur Überblick über einen kleinen Teil von EvoSuite geben.

\section{Genetische Algorithmen}
\label{sec:genetische_algorithmen}

Genetische Algorithmen gehören zu den Evolutionären Algorithmen.
Die Idee dabei ist es, den Prozess der Evolution und natürlichen Auslese aus der Biologie zu simulieren, um Optimierungsprobleme zu lösen.
Dazu werden Lösungskandidaten für das Optimierungsproblem als Individuen betrachtet.
Die Menge der betrachteten Individuen nennt man auch Population oder Generation.

Ausgehend von einer Anfangspopulation werden mit einer bestimmten Wahrscheinlichkeit durch Rekombination und/oder Mutation neue Individuen erzeugt.
Bei einer Rekombination werden aus mehreren Eltern-Individuen ein oder mehrere neue Kind-Individuen erzeugt, während bei einer Mutation aus nur einem Individuum durch eine kleine Änderung ein neues Individuum entsteht.
Die neuen Individuen werden anhand einer Fitnessfunktion bewertet und anhand von dieser Bewertung wird eine Menge der Individuen ausgewählt, um die neue Population zu bilden.
Die restlichen Individuen werden verworfen.

Dieser iterative Prozess wird so lange ausgeführt, bis ein Individuum gefunden wurde, welches das Optimierungsproblem optimal löst, bis eine vorgegebene Anzahl an Generationen entwickelt wurde, oder bis die verfügbare Zeit aufgebraucht ist.
In jedem Fall ist das Ergebnis des genetischen Algorithmus das beste Individuum.

Um ein konkretes Optimierungsproblem mit einem genetischen Algorithmus zu lösen, muss genau definiert werden: was ein Individuum ist, aus welchen Individuen die Anfangspopulation besteht, wie die Rekombinations- und Mutationsoperatoren neue Individuen erzeugen und wie die Fitnessfunktion aussieht.
Dies ist von Problem zu Problem unterschiedlich.

\section{Erzeugen von Testfällen}
\label{sec:erzeugen_von_testfaellen}

\subsection{Betrachten von ganzen Testsuites auf einmal}

EvoSuite beginnt mit dem Generieren von Testfällen ohne Orakel.
Dafür werden verschiedene Algorithmen angeboten, in dieser Seminararbeit wird jedoch nur das Verbessern von ganzen Testsuites hinsichtlich der Testabdeckung mit einem genetischen Algorithmus nach \cite{TSE12_EvoSuite} beschrieben.

Die Verbesserung von ganzen Testsuites auf einmal ist dabei eine wichtige Neuerung, die vor EvoSuite, nach bestem Wissen der Autoren, nicht verwendet wurde.
Nach \cite{TSE12_EvoSuite} betrachteten die meisten anderen Werkzeuge nur einzelne Programmpfade auf einmal und versuchten einzelne Testfälle zu generieren, die die ausgewählten Pfade abdeckten.
Ein großes Problem bei diesem Ansatz ist jedoch, dass nicht jeder Programmpfad auch erreichbar ist.
Wenn versucht wird einen unerreichbaren Pfad zu erreichen, dann ist der dafür verwendete Aufwand verschwendet.
Ob ein bestimmter Pfad überhaupt erreichbar ist, ist allerdings ein unentscheidbares Problem, da man sonst das Halteproblem lösen könnte.
Bei diesem Ansatz ist es demnach sehr wichtig, mit welchen Programmpfaden begonnen wird.

Ein weiteres Problem ist Kollateralabdeckung.
Wenn zunächst Testfälle generiert werden, die einfache Pfade nahe der Oberfläche des Programms abdecken, kann es passieren, dass dieselben Pfade auch erreicht werden, wenn später Testfälle für tiefere Pfade des Programms generiert werden.
Dadurch werden die zuerst generierten Testfälle nutzlos, da sie keinen Beitrag mehr zur Testabdeckung beitragen und der Aufwand sie zu generieren war verschwendet.

EvoSuite vermeidet diese Probleme, indem die Testabdeckung der gesamten Testsuite auf einmal verbessert wird.
Statt einzelne Programmpfade auszuwählen, die erreicht werden sollen, wird bei EvoSuite gemessen, wie weit die gesamte Testsuite davon entfernt ist, alle Programmpfade abzudecken.

\subsection{Testsuites als Individuen}

Wie in \cref{sec:genetische_algorithmen} erwähnt, muss für die Nutzung eines genetischen Algorithmus zunächst definiert werden, wie ein Individuum aussieht.
Da EvoSuite ganze Testsuites auf einmal betrachtet, ist ein Individuum eine Testsuite.
Eine Testsuite ist dabei eine Menge von Testfällen ($T = \{t_1, t_2, ..., t_n\}$), wobei ein Testfall eine Sequenz von Anweisungen ist ($t = \langle s_1, s_2, ..., s_l \rangle$).
In EvoSuite wird zwischen 5 Arten von Anweisungen unterschieden:
\begin{itemize}
	\item Primitive Anweisungen, wie \lstinline{int var0 = 54},\\
	oder \lstinline{Object[] var1 = new Object[10]}.
	\item Zuweisungen, wie \lstinline{var1[0] = new Object()},\\
	oder \lstinline{var2.maxSize = 10}.
	\item Feld-Anweisungen, wie \lstinline{int var3 = var2.size}.
	\item Konstruktor-Anweisungen, wie \lstinline{Stack var2 = new Stack()}.
	\item Methoden-Anweisungen, wie \lstinline{int var4 = var2.pop()}.
\end{itemize}
Der Sprachumfang von Java erlaubt zwar deutlich mehr Arten von Anweisungen, wie z. B. \lstinline{if}-Anweisungen, \lstinline{while}- und \lstinline{for}-Schleifen, diese sind jedoch nicht in Testfällen erwünscht, da Testfälle deterministisch sein sollen und keine Parameter erhalten, die sich auf den Kontrollfluss auswirken könnten.
Außerdem enthalten die hier definierten Testfälle keine Orakel, wie z. B. \textit{Assertions}.
Diese werden in einem späteren Schritt hinzugefügt (Siehe \cref{sec:erzeugen_von_orakeln}).

Für den noch zu definierenden Mutationsoperator ist es wichtig festzuhalten, dass jede Anweisung einen Wert von einem bestimmten Typ bereitstellt, der in späteren Anweisungen verwendet werden kann.
Für eine Anweisung $s$ wird dieser Wert im folgenden als $v(s)$ bezeichnet.

\subsection{Erzeugen neuer Individuen}

Als nächstes müssen die Rekombination und Mutation definiert werden.
Die Rekombination ist in EvoSuite sehr einfach gehalten.
Hierbei werden mit einer bestimmten Wahrscheinlichkeit aus den beiden Eltern-Testsuites $P_1$ und $P_2$ die beiden Kind-Testsuites $O_1$ und $O_2$ erstellt.
Zunächst wird ein Anteil $\alpha \in [0;1]$ gewählt.
Das Kind $O_1$ ergibt sich, indem der Anteil $\alpha$ der Testfälle von $P_1$ mit einem Anteil $1 - \alpha$ der Testfälle von $P_2$ kombiniert wird.
Das Kind $O_2$ besteht dann aus den übrigen $(1 - \alpha) \times |P_1|$ Testfällen von $P_1$ und den übrigen $\alpha \times |P_2|$ Testfällen von $P_2$.
Individuen können durch diese Rekombination nicht länger oder kürzer werden, als der längste, bzw. kürzeste Elternteil.
Im Durchschnitt nähern sich dadurch alle Individuen der gleichen Anzahl an Testfällen an.

Die Mutation einer Testsuite $T$ ist deutlich komplexer, als die Rekombination.
Hierbei wird jeder Testfall $t$ mit einer Wahrscheinlichkeit von $\frac{1}{|T|}$ mutiert.
Im Durchschnitt wird bei der Mutation einer Testsuite also genau ein Testfall mutiert.
Es ist aber auch möglich, dass keiner oder mehrere Testfälle auf einmal mutiert werden. Bei der Mutation eines Testfalls werden der Reihe nach folgende Operationen jeweils mit einer Wahrscheinlichkeit von $\frac{1}{3}$ durchgefüht:
\begin{enumerate}
	\item Entfernen von Anweisungen.
	\item Ändern von Anweisungen.
	\item Einfügen von Anweisungen.
\end{enumerate}

Beim Entfernen von Anweisungen wird jede Anweisung $s_i$ mit einer Wahrscheinlichkeit von $\frac{1}{|t|}$ entfernt.
Da der Wert $v(s_i)$ in einer späteren Anweisung benutzt werden könnte, müssen alle Anweisungen $\{s_j \in t | j > i\}$ überprüft werden, um sicherzustellen, dass ein valides Individuum entsteht.
Wenn ein $s_j$ den Wert $v(s_i)$ verwendet, dann wird dieser durch einen anderen verfügbaren Wert desselben Typs ersetzt.
Wenn in dem Testfall keine andere Anweisung einen Wert mit demselben Typ bereitstellt, dann wird $s_j$ rekursiv entfernt.
Bleibt nach dem Entfernen von Anweisungen ein leerer Testfall übrig, dann wird dieser komplett gelöscht.

Beim Ändern von Anweisungen, wird wieder jede Anweisung mit einer Wahrscheinlichkeit von $\frac{1}{|t|}$ geändert.
Hier werden verschiedene Arten von Anweisungen verschieden behandelt.

Bei Primitiven Anweisungen wird einfach der primitive Wert um einen zufälligen $\Delta$-Wert erhöht oder erniedrigt, Arrays werden verlängert oder verkürzt und Zeichenketten werden mutiert.
Bei der Verkürzung von Arrays muss genau wie beim Entfernen von Anweisungen darauf geachtet werden, dass spätere Anweisungen keine entfallenen Arrayindizies verwenden.
Bei der Mutation von Zeichenketten werden, ähnlich wie bei der Mutation von Testfällen, neue Zeichen eingefügt, bestehende Zeichen geändert, und/oder alte Zeichen entfernt.

Bei Zuweisungen wird zufällig eine der beiden Seiten durch ein anderes Feld, bzw. einen anderen Wert ersetzt.

Für Feld-, Konstruktor- und Methoden-Anweisungen wird zufällig ein anderes Feld, ein anderer Konstruktor, oder eine andere Methode ausgesucht, deren Parameter mit den verfügbaren Werten erfüllt werden können.
Dabei ist es z. B. auch möglich, dass aus einer Feld-Anweisung eine Methoden-Anweisung wird.

Beim Einfügen von Anweisungen wird dem Testfall mit einer Wahrscheinlichkeit von $\sigma$ eine zufällige Anweisung hinzugefügt.
Als Parameter für Methoden- oder Konstruktoraufrufe werden dabei die bereitgestellten Werte wiederverwendet oder neue Werte generiert, die direkt eingesetzt werden.
Dies wird wiederholt, bis in einem Schritt mit einer Wahrscheinlichkeit von $1 - \sigma$ keine neue Anweisung eingefügt wird.
Durch dieses Vorgehen sinkt die Wahrscheinlichkeit weitere Anweisungen einzufügen exponentiell.
$i$ oder mehr Anweisungen werden also nur mit einer Wahrscheinlichkeit von $\sigma^i$ eingefügt.

Mit dem bis hier beschriebenen Mutationsoperator können Testfälle länger und kürzer werden und sogar entfallen.
Es ist aber bisher nicht möglich, dass neue Testfälle entstehen und auch durch die Rekombination kann die Anzahl der Testfälle in einer Testsuite nicht größer werden, als die Anzahl der Testfälle im größeren Elternteil.
Aus diesem Grund wird bei der Mutation einer Testsuite nach dem Mutieren der Testfälle mit einer Wahrscheinlichkeit von $\sigma'$ ein zufälliger neuer Testfall erzeugt.
Genau wie bei dem Einfügen von Anweisungen werden hier mit exponentiell sinkender Wahrscheinlichkeit weitere Testfälle hinzugefügt.

Um einen Testfall zu Erzeugen, wird eine zufällig Länge $l \sim [1, L]$ gewählt (wobei $L$ die maximal zugelassene Länge von Testfällen ist) und es werden so lange neue Anweisungen, wie bei der Mutation von Testfällen, eingefügt, bis der neue Testfall die gewünschte Länge erreicht hat.

Nun da die Mutation von Testsuites definiert ist, ist es leicht die Anfangspopulation zu definieren.
Als Anfangspopulation wird eine vorgegebene Anzahl an Testsuites erzeugt, indem jede mit einer vorgegeben Anzahl an zufälligen Testfällen befüllt wird.
Die zufälligen Testfälle werden dabei genau wie bei der Mutation von Testsuites erstellt.

\subsection{Fitnessfunktion}

Als letztes muss für den genetischen Algorithmus noch die Fitnessfunktion definiert werden.
Hier implementiert EvoSuite verschiedene Ansätze.
Z. B. können Testfälle mit \textit{Mutation-Testing} generiert werden \cite{TSE12_Mutation}.
In dieser Arbeit wird jedoch die Maximierung der Testabdeckung aus \cite{TSE12_EvoSuite} betrachtet.
Gesucht ist dabei eine Testsuite, die mit möglichst wenig Anweisungen, möglichst viele Zweige der getesteten Klasse ausführt.
Jede Methode hat dabei einen Programmpfad (den Körper der Methode), welcher sich an einer \lstinline{if}-Anweisung in zwei Pfade aufteilt.
Diese Pfade können sich selbst wieder aufteilen oder wieder zu einem Pfad zusammenlaufen.
Da EvoSuite nicht Quellcode, sondern Bytecode betrachtet, sind kompliziertere Anweiungen, wie \lstinline{switch}-Anweisungen oder Schleifen bereits in einfache Verzweigungen übersetzt worden.
Eine optimale Testsuite führt demnach jede Anweisung in der getesteten Klasse aus, indem jede Bedingung mindestens einmal zu wahr und falsch evaluiert wird.

Da genetische Algorithmen durch Mutation nur kleine zufällige Änderungen erzeugen, ist es wichtig, dass bereits kleine Änderungen eine Verbesserung der Fitness erreichen können.
Wenn sich die Fitnessfunktion nur verbessert, wenn mehr Programmpfade abgedeckt werden, ist es sehr unwahrscheinlich, dass eine kleine Änderung die Fitness verbessert, da es sehr schwer ist genau die richtigen Werte zu erraten um eine Bedingung zu erfüllen.
Aus diesem Grund wird die \textit{branch distance} $d_{min}(b, T)$ von \citet{10.1109/32.57624} verwendet.

\begin{figure}[h]
	\begin{lstlisting}[basicstyle=\ttfamily\tiny]
public static int fakultaet(int n) {
  if (n > 16) { // Zweig b
    throw new IllegalArgumentException(n + "! is out of int range");
  }
  ...
}
	\end{lstlisting}
	\begin{multicols}{2}
		\begin{lstlisting}[basicstyle=\ttfamily\tiny]
@Test(timeout = 4000)
public void test01() throws Throwable {
  int int0 = fakultaet(1);
  assertEquals(1, int0);
}
		\end{lstlisting}
		\textit{branch distance} $\approx$ 16
		\columnbreak
		\begin{lstlisting}[basicstyle=\ttfamily\tiny]
@Test(timeout = 4000)
public void test10() throws Throwable {
  int int0 = fakultaet(10);
  assertEquals(3628800, int0);
}
		\end{lstlisting}
		\textit{branch distance} $\approx$ 7
	\end{multicols}
	\caption{Beispiel für \textit{branch distance}}
	\label{fig:branch_distance}
\end{figure}

Die Idee hierbei ist, dass eine Ausführung einer Bedingung besser ist, wenn die Variablen der Bedingung weniger geändert werden müssen, um es zu erfüllen.
\cref{fig:branch_distance} zeigt ein Beispiel, wie die \textit{branch distance} für einen Zweig \texttt{b} vom Wert einer Variable \texttt{n} abhängt.
Wenn die Variable \texttt{n} den Wert 1 hat, muss sie noch um 16 geändert werden, um die Bedingung zu erfüllen.
Wenn die Variable \texttt{n} den Wert 10 hat, muss sie nur noch um 7 geändert werden.
Demnach ist der rechte Testfall näher daran Zweig \texttt{b} abzudecken.
Die \textit{branch distance} einer Testsuite ist einfach das Minimum der \textit{branch distances} der einzelnen Testfälle.

Tatsächlich ist die Berechnung der \textit{branch distance} etwas komplizierter, so wird z. B. bei Bedingungen, die zu wahr evaluieren sollen noch eine Konstante $k > 0$ addiert, es müssen auch komplexe Bedingungen mit mehreren Variablen berücksichtigt werden und manche Bedingungen werden wegen Verschachtelung gar nicht ausgeführt, bevor andere Bedingungen erfolgreich sind.
Um die Fitnessfunktion von EvoSuite zu verstehen sind diese Details aber nicht wichtig.

Die Fitnessfunktion soll also alle Methoden ($M$) der getesteten Klasse ausführen und in den Methoden alle Zweige ($B$).
Für eine Testsuite $T$ ergibt sich damit die zu minimierende Fitnessfunktion in \cref{fig:fitnessfunktion}.
$M_T$ ist dabei die Menge der Methoden der getesteten Klasse, die durch die Testsuite abgedeckt werden und die Funktion $v(x)$ ist eine Normalisierungsfunktion um die \textit{branch distance} in einen Wertebereich von 0 bis 1 zu übersetzen.

\begin{figure}[h]
	$fitness(T) = |M| - |M_T| + \sum\limits_{b \in B} d(b, T)$

	$d(b, T) = \begin{cases}
	0 & \text{if the branch b has been covered,}\\
	v(d_{min}(b, T)) & \text{if the predicate was executed at least twice,}\\
	1 & \text{otherwise.}
	\end{cases}$

	$v(x) = \frac{x}{(x+1)}$
	\caption{Fitnessfunktion aus \cite{TSE12_EvoSuite}}
	\label{fig:fitnessfunktion}
\end{figure}

Wie man sieht wird die \textit{branch distance} $d_{min}(b, T)$ von \citet{10.1109/32.57624} nur für Zweige verwendet, die mindestens zweimal ausgeführt werden.
Der Grund dafür ist, dass es für jede Bedingung zwei Zweige gibt.
Wenn eine Bedingung durch eine Testsuite nur einmal ausgeführt wird, ist die \textit{branch distance} nur für einen der beiden Zweige 0.
Eine Minimierung der \textit{branch distance} für den anderen Zweig würde aber dazu führen, dass der erste Zweig nicht mehr abgedeckt wird, da die Bedingung bei nur einer Ausführung nicht wahr und falsch sein kann.
Nun würde die Fitnessfunktion wiederum Individuen bevorzugen, die die \textit{branch distance} des ersten Zweigs minimieren und es ergibt sich ein oszillierendes Verhalten.
Dieses Problem wird durch die Fallunterscheidung in der Funktion $d(b, T)$ vollständig vermieden.

\section{Erzeugen von Orakeln mit Mutation-Testing}
\label{sec:erzeugen_von_orakeln}

Mit dem Vorgehen in \cref{sec:erzeugen_von_testfaellen} können Testsuites mit möglichst hoher Testabdeckung erzeugt werden, aber diese Testsuites enthalten noch keine Orakel.

Dieser Abschnitt stützt sich vor allem auf \cite{TSE12_Mutation}.

Mutation bedeutet hier etwas anderes, als bei genetischen Algorithmen.

\section{Search-Based Software Testing Competition}

EvoSuite hat bisher bei allen Seach-Based Software Testing (SBST) Competitions teilgenommen.
Dieser Wettbewerb wird seit 2013 jedes Jahr vom International Workshop on Search-Based Software Testing veranstaltet.
EvoSuite konnte sich dabei gegenüber den anderen Werkzeugen durchsetzen und erzielte in den Jahren 2013 und 2016 - 2019 jeweils den ersten Platz \cite{6571663, 7810701, 7967958, 8452806, 8812209}.
Im Jahr 2015 belegte EvoSuite den zweiten Platz \cite{7173585}.
Die Ergebnisse des Wettbewerbs im Jahr 2014 konnte ich leider nicht finden.

\section{Praxistauglichkeit}

\section{Evaluation}

EvoSuite ist wird öffentlich auf GitHub gehostet, wodurch der Quellcode einfach mit Git heruntergeladen werden konnte.
Mit einer Maven 3.6.2 Installation (\url{https://maven.apache.org/}) konnten ich leicht eigene Versionen von EvoSuite basierend auf dem \textit{master-Branch} und \textit{1.0.6-Tag} bauen.
Leider schlugen einige Tests im \textit{evosuite-client}-Projekt fehl, daher musste ich Maven mit \texttt{mvn install -DskipTests} aufrufen.

Da EvoSuite Testfälle generiert, die von einem Tester evaluiert werden müssen, um Fehler zu finden, wäre es für diese Seminararbeit zu aufwändig EvoSuite an einer echten Bibliothek zu evaluieren.
Stattdessen habe ich ein paar sehr einfache Klassen geschrieben und aus diesen mit EvoSuite Tests generiert.

EvoSuite kann per Kommandozeile aufgerufen werden und unterstützt eine große Anzahl and Parametern.
Da die kleinen Beispiele, die ich verwendete, keine externen Bibliotheken benutzen, war benötigte ich für diese Beispiele nur den \texttt{-target} Parameter.
Ein Aufruf von EvoSuite sieht damit wie folgt aus: \texttt{java -jar evosuite-master-1.0.6.jar -target classesDir}

\section{Fazit}

\bibliography{references}
\bibliographystyle{plainnat}

\end{document}

%Originalpapiere siehe z.B.: \cite{TSE12_EvoSuite,10.1109/32.57624,emse14_mutation},
%Zitieren mit Autorennamen, z.B.: \citet{TSE12_EvoSuite}.
%Grundlagen in \cref{sec:background} und Code Beispiel in \cref{fig:abs}.
%
%Infos zu \LaTeX: \url{https://en.wikibooks.org/wiki/LaTeX}
%
%\begin{figure}
%\centering
%\begin{verbatim}
%int abs(int x) {
%    if(x >= 0) return x;
%    else return -x;
%}
%\end{verbatim}
%\caption{Betragsfunktion in C/Java}
%\label{fig:abs}
%\end{figure}
