\documentclass[a4paper,11pt]{article}

% Macht evtl unter Windows Probleme?
\usepackage[utf8]{inputenc}

% weitere Pakete hier

% Referenzen schön formatieren, Kommandos
% \cite  -> [x]
% \citet -> Name [x]
\usepackage[sort&compress,numbers]{natbib}
\setlength{\bibsep}{2pt plus 0.3ex}
\renewcommand*{\bibfont}{\small}
\makeatletter
\def\NAT@spacechar{~}% NEW
\makeatother

% Klickbare Links im Dokument
\usepackage{hyperref} % muss als vorletztes Paket eingebunden werden
% Evtl auskommentieren

% Querverweise mit \cref und \Cref
\usepackage[ngerman]{babel}
\usepackage[ngerman,noabbrev]{cleveref} % muss als letztes Paket eingebunden werden


%%%%%%%%%%%%%%%%%%%%%%%%%%%%%%%%%%%%%%%%
\author{Adrian Uffmann\\Matrikelnummer: 12043921\\\texttt{adrian.uffmann@campus.lmu.de}}
\title{\textbf{EvoSuite}\\Automatic Test Suite Generation for Java}
%%%%%%%%%%%%%%%%%%%%%%%%%%%%%%%%%%%%%%%%

\begin{document}
\maketitle

\begin{center}
Masterseminar Fuzz Testing
\end{center}

\begin{abstract}
In dieser Seminararbeit wird X besprochen.
X löst Problem Y. Die Hauptidee bei Y ist Z.
X ist ziemlich cool.
\end{abstract}

\section{Einleitung}
\label{sec:intro}

Originalpapiere siehe z.B.: \cite{TSE12_EvoSuite,10.1109/32.57624,emse14_mutation},
Zitieren mit Autorennamen, z.B.: \citet{TSE12_EvoSuite}.
Grundlagen in \cref{sec:background} und Code Beispiel in \cref{fig:abs}.

Infos zu \LaTeX: \url{https://en.wikibooks.org/wiki/LaTeX}

\section{Grundlagen}
\label{sec:background}

\begin{figure}
\centering
\begin{verbatim}
int abs(int x) {
    if(x >= 0) return x;
    else return -x;
}
\end{verbatim}
\caption{Betragsfunktion in C/Java}
\label{fig:abs}
\end{figure}

\bibliography{references}
\bibliographystyle{plainnat}

\end{document}
